\documentclass{article}

\usepackage[a4paper, left=3cm, right=3cm, top=4cm]{geometry}
\usepackage{dirtree}


\title{Projekt-Dokumentation - Lernprogramm - Beleg Internettechnologien I}
\date{27-05-2024}
\author{Lotte Richter\\ s85482\\Matrikelnummer: 53042}

\begin{document}
  \maketitle
  \tableofcontents
  \newpage

\section{Projektübersicht}

Dieses Projekt ist eine Progressive Web App(PWA) für ein Lernprogramm, welches auch offlinefähig ist.
offlinefähig.\\\\
Das Programm ist ein Lernprogramm, dessen Funktionalität darin besteht, Fragen zu beantworten und damit einen Lernfortschritt zu erreichen.
Dazu kann man aus verschiedenen Kategorien wählen aus denen zufällige Fragen, passend zur Kategorie, gewählt werden.

Zu jeder Frage gibt es vier Antworten. Je nachdem, ob man die Frage richtig beantwortet, färbt sich der Button der Antwort rot oder grün ein
und die entsprechende Progress-Bar wird erhöht. Ist die Antwort falsch, so kann man sie direkt erneut beantworten. Ist die Antwort richtig, so wird
eine neue Frage geladen.

Wenn eine der beiden Progress-Bars 100\% erreicht, wird eine kurze Auswertung angezeigt.

\section{Verzeichnisstruktur}
Das Projekt ist folgendermaßen strukturiert:\\

\dirtree{%
  .1 Lernprogramm.
    .2 doc.
      .3 Projekt-Dokumentation\_Richter\_Lotte.tex.
      .3 Projekt-Dokumentation\_Richter\_Lotte.pdf.
    .2 images.
      .3 demo.png.
    .2 scripts.
      .3 sw.js.
    .2 src.
      .3 lernprogramm.css.
      .3 lernprogramm.html.
      .3 lernprogramm.js.
      .3 manifest.json.
    .2 Beleg-Aufgabenformat.md.
    .2 Beleg-Aufgabenstellung.md.
    .2 README.md.
}

\section{Technologien und Abhängigkeiten}

\begin{itemize}
  \item HTML5: Struktur der Webanwendung
  \item CSS3: Styling der Anwendung
  \item JavaScript: Funktionalität der Anwendung
  \item Service Worker: Offline-Fähigkeit und Caching
  \item Fetch API: Abrufen von Quizfragen von einem externen Server
  \item PWA Manifest: Definition der PWA-Eigenschaften
  \item KaTeX: Rendering von mathematischen Formeln
\end{itemize}
Bei dem Entwurf des Grundgerüsts des Programmes habe ich mich an den Beispielen aus der Vorlesung und Praktikum orientiert.\\\\
Um den Quizserver nutzen zu können, muss man sich im Netz der HTW Dresden befinden, da sonst der Server nicht erreicht werden
kann, von dem die Fragen geholt werden.

\section{Weiterentwicklung}
Konfigurationen:
\begin{itemize}
  \item Es können mehr Fragen vom Quizserver geholt werden, wenn man die Konstante SERVER\_QUESTIONS in der Datei lernprogramm.js erhöht.
  Die Konstante gibt an, wie viele Pages vom Server geholt werden. Dabei besitzt eine Page zehn Fragen.
  \item Es kann die Anzahl an beantworteten Fragen geänder werden, die es braucht bis eine Progress Bar 100\% erreicht. Dazu wird die Konstante
  TOTAL\_QUESTIONS in der Datei lernprogramm.js verwendet.
  \item Die CSS-Formatierung der Progress Bars funktioniert nur mit dem Browser Firefox. Wenn man eine anderen Browser nutzt, enpfiehlt es sich
  die im Code markierten Stellen zu ändern.
\end{itemize}
Erweiterung des Fragenkatalogs:
\begin{itemize}
  \item Weitere Fragen können in den jeweiligen Kategorien im 'questions' Objekt hinzugefügt werden.
\end{itemize}
UI-Verbesserungen:
\begin{itemize}
  \item Man kann die CSS-Datei anpassen, um die Benutzeroberfläche des Lernprogramms zu verbessern.
\end{itemize}
Button Highlight der Kategorie Quizserver
\begin{itemize}
  \item Nach dem Anklicken eines Buttons wird dieser rot oder grün hervorgehoben, je nachdem ob die Antwort richtig ist. Da beim Quizserver die Antworten
  anders geprüft werden, als bei den lokalen Fragen und das ganze asynchron läuft, ist mir keine elegante Lösung zur Hervorhebung der Buttons gefunden.
\end{itemize}

\section{Nutzung von KI}
\subsection{ChatGPT}
Zum Entwicklen des Programmes wurd an manchen Stellen die KI "ChatGPT" genutzt. Hauptsächlich zu Beginn des Projektes, sobald das Projekt und 
die Quellcodes ein bisschen umfangreicher wurde, war die KI nicht mehr so hilfreich. 

Genutzt wurde ChatGPT bei simplen Aufgaben, wie das Einlesen der Daten aus der JSON-Datei, oder bei der Erstellung von Dokumenten, wie. z.B. die
Projekt-Dokumentation.

\end{document}